%% Template for ENG 401 reports
%% by Robin Turner
%% Adapted from the IEEE peer review template

%
% note that the "draftcls" or "draftclsnofoot", not "draft", option
% should be used if it is desired that the figures are to be displayed in
% draft mode.

\documentclass{article}
\usepackage{url} % Provides better formatting of URLs.
\usepackage[utf8]{inputenc} % Allows Spanish characters.
\usepackage[spanish]{babel}
\usepackage{amssymb}
\usepackage{booktabs} % Allows the use of \toprule, \midrule and \bottomrule in tables for horizontal lines
\usepackage{float}
\usepackage{graphicx}
\usepackage{booktabs}
\usepackage{fancyhdr}
\usepackage{eurosym}


\hyphenation{op-tical net-works semi-conduc-tor} % Corrects some bad hyphenation 



\begin{document}
%\begin{titlepage}
% paper title
% can use linebreaks \\ within to get better formatting as desired
\title{\includegraphics[scale=0.65]{logoDefinitivo1.png} \\ \textbf{Proyecto UPBEAT} \\ \textbf{Grupo 03. Barbara Liskov}\\Plan de gestión, análisis y memoria del proyecto \vspace{0.1cm} \\}

% author names and affiliations

\author{Alejandro Ruiz Sumelzo\\
Alejandro Piedrafita Barrantes\\
Álvaro Santamaría De la Fuente\\
Fernando Navarro Zarralanga\\
José Félix Yagüe Royo\\
Víctor Pérez Sanmartín\\
Sergio Torres Castillo \vspace{0.25cm}
\\\\
\includegraphics[scale=0.5]{logoUZ.png}\\
}
\date{24 de febrero de 2020}

% make the title area
\maketitle
\newpage
\section*{Resumen ejecutivo}
En este documento se detalla la propuesta técnica y económica del proyecto UPBEAT.\\  \hfill \break
Este proyecto tiene como objetivo la creación de un servicio de streaming de música, con las funcionalidades básicas de un reproductor, para canciones y podcasts, con opciones para modificar el sonido, aspectos de red social que permiten añadir amigos y el uso de banners para poder patrocinar productos. \\
Además existirá un perfil de artista con más funcionalidades que el resto de usuarios, que permitirán gestionar las canciones creadas y/o producidas por los mismos.\\

Esta aplicación podrá ser accedida o bien mediante un cliente web o bien desde una aplicación móvil, disponible tanto para IOS como para Android. El servidor contará con una API REST para intercambiar datos entre el servidor y los clientes.\\

En cuanto a la descripción técnica del proyecto, se especifica que el sistema se desarrollará sobre una capa de aplicación web basada en Angular, cumplimentado con una app móvil la cual será creada utilizando Flutter, aplicación usada para el desarrollo de interfaces de usuario para aplicaciones móviles que permite la compilación para ambos sistemas operativos.\\
Esto permitirá usar servicios REST, con Spring como herramienta de desarrollo, aplicación que ya ha sido utilizada por uno de los miembros de este proyecto. Para la base de datos, se ha decidido utilizar PostgreSQL como la mejor alternativa tras comparar diferentes opciones y reunirse con el profesor asignado. \\ 

Para todo ello, el plan de trabajo que se ha cumplimentado y acordado ha sido establecido por el cliente, contando éste con dos iteraciones en las cuales se deberán ir mostrando los avances logrados y los requisitos que se han satisfecho hasta la fecha.
El equipo técnico está formado por siete estudiantes de la Universidad de Zaragoza, los cuales han sido formados y siguen formándose sobre conocimientos de programación, bases de datos, sistemas de la información y administración de sistemas entre otras competencias adquiridas, con experiencia en diversos proyectos llevados a cabo a lo largo de las diferentes asignaturas que permitirán superar las diversas fases del proyecto a pesar de la escasa experiencia laboral.\\

Por último, se adjunta también el presupuesto total de la inversión para el proyecto UPBEAT, con el desglose del mismo y los diferentes costes que supone.

\newpage
\tableofcontents % si estas líneas se comentan, se eliminan los índices
%\listoffigures
%\listoftables
\addtocontents{toc}{\hfill \textbf{Página} \par}
\newpage
\pagestyle{fancy}
\lhead{\begin{picture}(0,0) \put(0,0){\includegraphics[width=40mm]{logoEina.png}} \end{picture}}
\rhead{\begin{picture}(0,0) \put(-100.7,0){\includegraphics[width=35mm]{logoDefinitivo3.png}} \end{picture}}
\section{Objetivos del sistema}

\subsection{Análisis de requisitos preliminar}
Se exponen los siguientes requisitos funcionales de la aplicación:
\begin{table}[H]
	\begin{tabular}{p{4cm} p{10cm}}
		\hline
		\hline 
		\textbf{Requisito funcional}
		\vspace{0.5mm} & \textbf{Descripción} \\ 
		\hline
		\hline
		RF1
		& El sistema permite almacenar canciones y podcasts en formato \textit{.mp3}, \textit{.WAV} y \textit{.WMA}. \\ 
		\hline 
		RF2
		& El usuario accede al sistema mediante una aplicación móvil o una aplicación web. \\ 
		\hline
		RF3
		& El artista accede al sistema mediante una aplicación web. \\ 
		\hline
		RF4
		& El sistema permite reproducir y pausar una canción. También permite saltar a la siguiente (si la hubiera), retroceder a la anterior, y elegir un bucle de la misma o reproducir aleatoriamente varias canciones. \\ 
		\hline
		RF5
		& El sistema permite al usuario registrado subir o bajar el volumen de la canción/podcast en reproducción. \\ 
		\hline
		RF6
		& El usuario debe registrarse en el sistema o iniciar sesión para acceder a sus funcionalidades. \\ 
		\hline
		RF7
		& El usuario y el artista acceden al sistema mediante un nombre identificatorio y una contraseña. \\ 
		\hline
		RF8
		& El usuario registrado puede buscar canciones, álbumes, artistas y podcasts. \\ 
		\hline
		RF9
		& El usuario registrado puede crear playlists (agrupación de una o varias canciones) . \\ 
		\hline
		RF10
		& El usuario registrado puede añadir/eliminar de sus favoritos una canción, playlist, álbum o podcast . \\ 
		\hline
		RF11
		& El usuario registrado puede ordenar por fecha de lanzamiento, nombre y por artista las canciones añadidas a una playlist. \\ 
		\hline
		RF12
		&  El usuario registrado puede seguir a otros usuarios dentro del sistema.\\
		\hline
		RF13
		&  El usuario registrado puede filtrar la búsqueda de una determinada canción por género y época.\\
		\hline
		RF14
		&  El sistema permite sincronizar varios dispositivos, de tal manera que si reproduces la misma canción con el mismo usuario registrado en distintos dispositivos, solo puedas hacerlo en uno de ellos.\\
		\hline
		RF15
		&  El sistema permitirá tener \textit{banners} de anuncios en la aplicación web.\\
		\hline
		RF16
		& El sistema permite al usuario registrado manejar un ecualizador del sonido en la aplicación web. \\ 
		\hline
		RF17
		& El artista tiene todas las funcionalidades del usuario registrado, además de poder subir canciones y podcast. \\ 
		\hline
		RF18
		& El artista puede elegir la fecha y hora de una publicación, o subirla inmediatamente. \\ 
		\hline
	\end{tabular}
\end{table}
\break
Se exponen los siguientes requisitos no funcionales de la aplicación:

\begin{table}[H]
	\begin{tabular}{p{4cm} p{10cm}}
		\hline
		\hline 
		\textbf{Requisito no funcional} & \textbf{Descripción} \\ 
		\hline
		\hline
		RNF1 
		&  El sistema permitirá ser utilizar un diseño modular, un lenguaje fácil de entender, usar y mantener.\\ 
		\hline
		RNF2
		&  La seguridad será fundamental a la hora de garantizar la confidencialidad y autentificación de las canciones, así como para cumplir determinados aspectos de la LOPD y los derechos de los cantautores.\\ 
		\hline
		RNF3
		&  El cliente tendrá el desarrollo móvil en formato Android e iOS.\\ 
		\hline
	\end{tabular}
\end{table}
\newpage

\section{Descripción técnica}

\subsection{Aspectos técnicos para el usuario}
El usuario solamente necesitará de un navegador web para conseguir conectarse al proyecto \textit{UPBEAT}, además de usar aplicaciones para Android e iOS.
Este proyecto será compatible con el navegador \textit{Google Chrome}.\vspace{0.5cm}
\hfill\break
\begin{figure}[H]

\end{figure}

\subsection{Aspectos técnicos para el cliente}
El proyecto será entregado en un servidor web de hosting, el cual implementará todos los servicios necesarios para que funcione correctamente. Se le proporcionará un usuario y contraseña para acceder al mismo y hacerse cargo una vez se desarrolle y entregue el proyecto.
\hfill \break
También se le proporcionará al cliente los archivos de ambas aplicaciones desarrolladas para la tecnología móvil.
\begin{figure}[H]

\end{figure}

\newpage
\subsection{Descripción técnica preliminar}
Para la realización del proyecto, el sistema usará la siguiente estructura: \vspace{0.15cm}

\begin{figure}[H]

\end{figure}
\vspace{0.15cm}

El desarrollo del cliente web se centrará en \textit{Angular} principalmente, el cual se conectará al servidor de aplicaciones que proporcionará \textit{Spring Boot}. Este último gestionará las consultas y conexiones con la base de datos, desarrollada e implementada en \textit{PostgreSQL}.
Para el desarrollo de la aplicación móvil se usará \textit{Flutter}:
\begin{figure}[H]

\end{figure}
\newpage

\section{Plan de trabajo}
\begin{itemize}
	\item ¿Qué se va a entregar?
	\begin{itemize}
		\item \textbf{Propuesta técnica y económica}: requisitos, prototipos de diseño, presupuesto y costes.
		\item \textbf{Plan de gestión, análisis, diseño y memoria del proyecto}: 2 iteraciones del proyecto
		\item \textbf{Prototipo funcional del sistema}, el cual va a ir evolucionando a medida que avanza el proyecto, con el objetivo de obtener feedback del cliente.
		\item \textbf{Manual de usuario}.
	\end{itemize}
\end{itemize}

\begin{itemize}
	\item El proyecto se va a dividir en 3 fases:
	\begin{itemize}
		
		\item \textbf{Definición e inicio del proyecto: hasta el 16/03/2020}
		\hfill \break
		2 entregas: \begin{itemize}
			\item Propuesta técnica y económica (24/02/2020)
			\item Plan de gestión, análisis, diseño y memoria del proyecto versión 1 (16/03/2020)
		\end{itemize}
		2 reuniones con profesor de seguimiento y apoyo.
		\item \textbf{Primera iteración del proyecto: hasta el 15/04/2020}
		\hfill \break
		1 entrega: \begin{itemize}
			\item Plan de gestión, análisis, diseño y memoria del proyecto versión 2 (15/04/2020)
		\end{itemize}
		2 reuniones:
		\begin{itemize}
			\item 1 reunión con profesor para seguimiento y apoyo
			\item 1 reunión con profesor para demostración del software intermedio
		\end{itemize}
		\item \textbf{Segunda iteración del proyecto: hasta el 08/06/2020}
		\hfill \break
		1 entrega: \begin{itemize}
			\item Plan de gestión, análisis, diseño y memoria del proyecto versión final + manual de usuario(08/06/2020)
		\end{itemize}
		2 reuniones:
		\begin{itemize}
			\item 1 reunión con profesor para seguimiento y apoyo
			\item 1 reunión con profesor para demostración del software final
		\end{itemize}
	\end{itemize}
\end{itemize}

\newpage

\section{Equipo técnico encargado del proyecto}
La empresa \textit{UPBEAT SA}, dedicada al desarrollo de proyectos de software por demanda, está compuesta por 7 estudiantes de la titulación de Ingeniería Informática en la Universidad de Zaragoza, los cuales cursan las ramas de sistemas de la información y tecnologías de la información.
\hfill \break
\\
Previamente, se han desarrollado proyectos relacionados con el trabajo a desempeñar, en los cuales se desarrolló e implementó una aplicación web de recomendación de automóviles de bajo consumo y de bonificaciones a las empresas por ser sostenibles y reducir las emisiones contaminantes en su actividad diaria. Para la realización de este trabajo se utilizó una arquitectura en 3 capas en las que se siguió un modelo vista-controlador.\\ Se trabajó con tecnologías como mariaDB, Docker para el despliegue de la aplicación o la utlización de html5, css o jsp y servlets de Java para el modelo estático y dinámico de la aplicación, respectivamente.
\hfill \break
\\
La empresa también cuenta con experiencia en varias tecnologías punteras en el mercado, como Docker, Angular y Spring Boot, los cuales han sido utilizados en proyectos de integración de logística empresarial.
La empresa en la que se ha trabajado con proyectos similares ha sido Carreras Grupo Logístico.
\hfill \break

Además uno de los integrantes del equipo, Sergio Torres, fue creador de un sitio web de comercio electrónico, el cual se encuentra actualmente cerrado, demostrando experiencia en la utlización de tecnologías de diseño de interfaces web como HTML5 o CSS.\\

La gran mayoría de los integrantes del equipo también realizaron ya una aplicación móvil de notas en la que se trabajó con Android Studio y se realizó un proyecto Software desde la fase de Análisis hasta el lanzamiento incluyendo la realización de pruebas del software realizado.\\

Además el equipo técnico conoce el proceso que se lleva a cabo en las metodologías ágiles, especialmente SCRUM y pizarras KANBAN, lo cual está directamente relacionado con el plan de trabajo acordado para este proyecto.
\newpage

\section{Presupuesto}
\textit{D. ALEJANDRO RUIZ SUMELZO}, domiciliado para estos efectos en Zaragoza, provincia de Zaragoza, calle \textit{Camino de Juslibol N.º 36}, y DNI \textit{25203767E} en nombre y representación de \textit{UPBEAT S.A.} con C.I.F. \textit{123456} y domicilio fiscal en Zaragoza, calle \textit{María de Luna}, edificio \textit{Ada Byron}, enterado del anuncio publicado en el Moodle de la Universidad de Zaragoza, a día 13 de febrero de 2020 y de las condiciones y requisitos que se exigen para la adjudicación de los servicios de la \textbf{REALIZACIÓN DE UNA APLICACIÓN DE REPRODUCCIÓN DE MÚSICA}, provincia de ZARAGOZA, se compromete a tomar a su cargo la ejecución de los mismos, con estricta sujeción a los requisitos y condiciones expresados, por las cantidades que se enumeran en concepto de precio, indicándose igualmente el IVA a soportar por la Administración, en cada caso, según las soluciones siguientes:

\begin{figure}[H]
\end{figure}

El licitador hace constar que el precio o precios del contrato ofertado incluye el importe de las tasas que sean de aplicación, con conformidad con lo señalado en el Pliego de Cláusulas Administrativas Particulares que rige este contrato. 
\newpage
\section*{Anexo I. Estimación de costes}
\begin{figure}[H]
\end{figure}

En total, los beneficios obtenidos serían de \textbf{14.226,69 \euro}.

\end{document}


