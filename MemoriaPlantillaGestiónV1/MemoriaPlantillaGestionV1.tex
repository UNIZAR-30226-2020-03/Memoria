%% Template for ENG 401 reports
%% by Robin Turner
%% Adapted from the IEEE peer review template

%
% note that the "draftcls" or "draftclsnofoot", not "draft", option
% should be used if it is desired that the figures are to be displayed in
% draft mode.

\documentclass{article}
\usepackage{url} % Provides better formatting of URLs.
\usepackage[utf8]{inputenc} % Allows Spanish characters.
\usepackage[spanish]{babel}
\usepackage{amssymb}
\usepackage{booktabs} % Allows the use of \toprule, \midrule and \bottomrule in tables for horizontal lines
\usepackage{float}
\usepackage{pdfpages}
\usepackage{graphicx}
\usepackage{booktabs}
\usepackage{fancyhdr}
\usepackage{eurosym}
\usepackage{calendar}



\hyphenation{op-tical net-works semi-conduc-tor} % Corrects some bad hyphenation 



\begin{document}
%\begin{titlepage}
% paper title
% can use linebreaks \\ within to get better formatting as desired
\title{\includegraphics[scale=0.65]{logoDefinitivo1.png} \\ \textbf{Proyecto UPBEAT} \\ \textbf{Grupo 03. Barbara Liskov}\\Plan de gestión, análisis y memoria del proyecto \vspace{0.1cm} \\}

% author names and affiliations

\author{Alejandro Ruiz Sumelzo\\
Alejandro Piedrafita Barrantes\\
Álvaro Santamaría De la Fuente\\
Fernando Navarro Zarralanga\\
José Félix Yagüe Royo\\
Víctor Pérez Sanmartín\\
Sergio Torres Castillo \vspace{0.25cm}
\\\\
\includegraphics[scale=0.5]{logoUZ.png}\\
}
\date{24 de febrero de 2020}

% make the title area
\maketitle
\newpage
\section*{Introducción}


\newpage
\tableofcontents % si estas líneas se comentan, se eliminan los índices
%\listoffigures
%\listoftables
\addtocontents{toc}{\hfill \textbf{Página} \par}
\newpage
\pagestyle{fancy}
\lhead{\begin{picture}(0,0) \put(0,0){\includegraphics[width=40mm]{logoEina.png}} \end{picture}}
\rhead{\begin{picture}(0,0) \put(-100.7,0){\includegraphics[width=35mm]{logoDefinitivo3.png}} \end{picture}}
\section{Organización del proyecto}
\begin{table}[H]
	\hspace*{-3.7cm}
	\centering
	\begin{tabular}{|l|l|l|}
		\hline
		\multicolumn{1}{|c|}{\textbf{Integrante}} & \multicolumn{1}{c|}{\textbf{Puesto}} & \multicolumn{1}{c|}{\textbf{Responsabilidades}}\\ \hline
		Alejandro Ruiz Sumelzo                    & \begin{tabular}[c]{@{}l@{}}Director del proyecto.\\ Coordinador y desarrollador del grupo de back-end.\\ Encargado de la documentación del análisis y diseño\\ del sistema.\end{tabular} & \begin{tabular}[c]{@{}l@{}}Responsable de redactar algunas actas\\ en reuniones con el profesor.\\ Control de la distribución de trabajo\\ (elaboración de calendario) y \\ revisión de esfuerzos.\\ Desarrollador de modelos, repositorios y \\controladores de la API.\\ Encargado del despliegue del back end \\sobre el servidor.\end{tabular} \\ \hline
		\begin{tabular}[c]{@{}c@{}}Alejandro\\ Piedrafita Barrantes\end{tabular}
		&                                                                       Desarrollador de apoyo para el grupo de back-end                                                                                                                 &  
		\begin{tabular}[c]{@{}l@{}}Realización de tareas de gestión\\  (edición de memoria y otros documentos).\\ Desarrollador de modelos, repositorios y \\ controladores de la API. \\ Diseño del sistema mediante diagramas.\end{tabular}\\ \hline
		\begin{tabular}[c]{@{}c@{}}Víctor\\ Pérez Santmartín\end{tabular}
		&                                                                       Desarrollador de apoyo para el grupo de back-end                                                                                                                 &  
		\begin{tabular}[c]{@{}l@{}}
		Responsable de redactar algunas actas\\ en reuniones con el profesor.\\
		Realización de tareas de gestión\\  (edición de memoria y otros documentos).\\ Desarrollador de modelos, repositorios y \\ controladores de la API. \\ Encargado del diseño e implementación\\ de la base de datos.\end{tabular}\\ \hline
	\end{tabular}
\end{table}

\section{Plan de gestión del proyecto}

\subsection{Procesos}

\subsubsection{Procesos de inicio del proyecto}

\subsubsection{Procesos de ejecución y control del proyecto}
El estándar a utilizar en el código, puesto que va ser una aplicación web, se utilizará TypeScript. Los responsables de realizar la puesta en marcha serán los encargados de la parte Front-end y de la parte Back-end. La creación de copias de seguridad y semejantes se realizarían de manera automática gracias a GitHub. 
El repositorio que se creará con todos los archivos referentes al proyecto se encontrará en GitHub, para que todos los integrantes del proyecto puedan acceder fácilmente a los archivos. Además, se usará el Issue Tracker de GitHub para la gestión de incidencias. 
El proyecto estará dividido en varios repositorios: uno específico para Front-end, otro para Back-end, y la memoria. Para conseguir que no se modifique el mismo fichero por dos personas al mismo tiempo y evitar problemas, cada equipo tendrá más sub-ramas de desarrollo, por ejemplo, una para cada miembro del equipo, que serán actualizadas con cambios no siempre funcionales y cuando sean más estables se volcarán a la rama de desarrollo principal. 
En la rama principal de cada uno de los repositorios sólo podrá haber una versión funcional del sistema, que antes de ser subida será sometida a diferentes test automáticos, entre los que se incluirán test para comprobar la estabilidad del sistema (pruebas de sobrecarga) y test que revisarán las acciones disponibles para comprobar los requisitos que se han resuelto. 
Para que lo desarrollado en cada uno de estos repositorios pase al repositorio funcional, cada líder de las respectivas partes revisará el código actualizado y si todo está correcto se considerará válido. Todos los componentes del equipo son capaces de modificar los ficheros de los repositorios excepto en el de las versiones, el cual solo podrán subir archivos y modificarlos los líderes del Front-End y el Back-End.

\subsubsection{Procesos técnicos}
\newpage
\subsection{Planes}

\subsubsection{Plan de gestión de configuraciones}
La convención de nombres utilizadas para nombrar los distintos archivos sería la siguiente: 
\begin{figure}[H]
	\centering{
		\includegraphics[scale=0.55]{conf1.png}}
\end{figure}
Las versiones solo se modificarán cada vez que se produzcan cambios suficientemente importantes, como por ejemplo la implementación de una nueva funcionalidad. 
Cada vez que se cree una nueva versión, pero sus cambios sean menores, como resolución de errores, se modificará su número de revisión, pero no de versión. 
Se crearán ficheros de documentación que permita ir recopilando toda la información referente a los cambios.
Además, en los ficheros de documentación en los que se expliquen las diversas funcionalidades que tiene la aplicación y que errores se han ido resolviendo, cuando estos sean de una nueva versión o revisión solo se ofrecerá la información sobre los cambios que existan entre esta y la versión o revisión anterior, pero siempre que se cambie la versión se documentarán los cambios respecto a la primera revisión de la versión anterior (p.ej. La versión 2.1 solo contendrá las novedades respecto a la versión 2.0, pero la versión 3.0 contendrá todos los cambios que hayan sucedido desde la versión 2.0 aunque la mayoría se hayan documentado ya en las revisiones). 

\newpage
\subsubsection{Plan de construcción y despliegue del software}

\subsubsection{Plan de aseguramiento de la calidad}

\newpage

\subsubsection{Calendario del proyecto y división del trabajo}
En la primera iteración del proceso de diseño nos centraremos en desarrollar las funcionalidades principales del sistema, mientras que en la segunda iteración se corregirán todos los errores encontrados en la primera, se implementarán las funcionalidades secundarias y se afinara el diseño de la página web y de las aplicaciones móviles para que sean más agradables al usuario. 
\hfill \break
Para la primera iteración, se planea permitir la creación, edición y borrado de clientes con sus credenciales básicos: nombre de usuario, nombre real, correo, contraseña. Unido a esto, comprobar si las entidades Artista y Usuario se crean y borrar correctamente.  También se permitirá la subida de canciones por parte de los artistas; estas canciones serán visibles en la aplicación y podrán ser reproducidas (al igual que los podcasts). Los álbumes estarán disponibles con su descripción y podrán ser consultados, reproduciendo cada una de sus canciones.
\hfill \break
Para la segunda iteración se finalizarán los requisitos que, por falta de tiempo, no pudieron ser completados en la primera y se añadirán funcionalidades al sistema. Estas funcionalidades son: añadir canciones a la lista de reproducción de un usuario, permitir información adicional en los perfiles de usuario (como puede ser una foto de perfil, una descripción, etc.), además de poder seguirse entre dos usuarios. Se permitirá la búsqueda y filtrado de determinadas canciones y/o álbumes por unos determinados parámetros, al igual que utilizar un ecualizador en la aplicación web con el uso de \textit{banners}.

\includepdf[pages=1-2]{calendar.pdf}
\newpage
\section{Análisis y diseño del sistema}

\subsection{Análisis de requisitos}

\subsection{Diseño del sistema}

\end{document}


